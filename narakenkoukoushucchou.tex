\documentclass[paper=a4paper,12pt]{jlreq}
\usepackage{luatexja}


%パッケージの読み込み
\usepackage{multirow}
\usepackage{array}
\usepackage[top=15truemm,left=10truemm,right=10truemm]{geometry}
\usepackage{tabularx}
\usepackage{arrayjobx}
\newarray\Data
\readarray{Data}%
{1gakkou&2namae&3bangou&4jusho&5小型\cdot フィット\cdot bangou&6nengappi&7nengappi&8nengappi&9\和暦\today&10youmu&11nenngappi&12jikan&13nengappi&14jikan&15gakkoujitaku&16jusho&17jusho&18jusho&19jusho&20jusho&21gakkoujitaku&22youmusaki&23jikan&24jikan&25chokkouchokki}

\renewcommand*{\arraystretch}{1.7}%セルの中身と線の間の空白
\linespread{0.5}\selectfont%セル内の行間
\pagestyle{empty}%ページ番号いらない 
\newcommand{\ctext}[1]{\raise0.2ex\hbox{\textcircled{\scriptsize{#1}}}}%丸印を綺麗に表示するためのもの\ctext{印}で表示される。
\begin{document}
%表部分スタート
\begin{table}[htbp]
	\centering
	{\Large 旅行伺兼旅行命令簿 (私有自動車等使用旅行用)}
	\vskip 0.2cm	
%所属、氏名等{{{
	\begin{tabularx}{\textwidth}{@{\vrule width1pt \ }>{\centering}m{2.2cm}|>{\centering\arraybackslash}X|>{\centering}m{4cm}|>{\centering\arraybackslash}X@{\ \vrule width1pt}}
\noalign{\hrule height 1pt}%太い線を引く方法
所  属&奈良県\Data(1)高等学校&職務の級(旅費級)&教育職 級\\
\hline
職・氏名&教諭 \Data(2) \ctext{印}&職 員 コ ー ド&\Data(3)\\
\hline
住 居 地&\multicolumn{3}{c@{\ \vrule width1pt}}{\Data(4)}\\
\noalign{\hrule height 1pt}%太い線を引く方法
\end{tabularx}
%所属、氏名等おわり}}}
\vskip 0.2cm
%車両情報{{{
\begin{tabularx}{\textwidth}{@{\vrule width1pt \ }>{\centering}m{3cm}|m{5cm}|>{\centering\arraybackslash}m{3cm}|m{6.355cm}@{\ \vrule width1pt}}
\noalign{\hrule height 1pt}%太い線を引く方法
{\footnotesize 車種\cdot 車名\cdot 車両番号}&{\footnotesize \Data(5)}&{\footnotesize 運転免許証 種類\newline\hfill 有効期限}&{\footnotesize 普通\newline \Data(6)}\\
%\noalign{\hrule height 1pt}%太い線を引く方法
\hline
{\footnotesize 車検証有効期限}&{\small \Data(7)}&{\footnotesize 任意保険 内容\newline\hfill 有効期限}&{\footnotesize 対人:無制限 対物:無制限\newline \Data(8)}\\
\noalign{\hrule height 1pt}%太い線を引く方法
\end{tabularx}
%車両情報おわり}}}
\vskip 0.2cm
%旅行期間{{{
\begin{tabularx}{\textwidth}{|>{\centering\arraybackslash}X|>{\centering\arraybackslash}X|>{\centering\arraybackslash}X|>{\centering\arraybackslash}X|>{\centering\arraybackslash}X|>{\centering}X|>{\centering}X|c|}%最後がXだとうまくいかない。
\hline
\multicolumn{5}{|c|}{\footnotesize 旅行命令及び私有自動車等使用承認決済欄}&\multirow{2}{*}{\scriptsize 宿泊数承認印}&\multirow{2}{*}{\scriptsize 命令変更確認印}&\multirow{2}{*}{\small 精算確認印}\\
\cline{1-5}
{\scriptsize 旅行命令権者}&{\small 教  頭}&{\small 事 務 長}&&&&&\\
\hline
 & & & & &要 泊& & \\
 & & & & &  & & \\
 & & & & &  & & \\
\hline
{\tiny 旅行命令年月日}&\multicolumn{2}{c|}{\Data(9)}&{\tiny 命令変更年月日}&\multicolumn{2}{c|}{令和 年 月 日}&入力済&\\
\noalign{\hrule height 1pt}%太い線を引く方法
\multicolumn{2}{@{\vrule width1pt \ }c|}{\multirow{2}{*}{旅 行 期 間}}&\multicolumn{2}{c|}{運転者職氏名}&\multicolumn{2}{c|}{同乗者職氏名}&\multicolumn{2}{c@{\ \vrule width1pt}}{用務}\\
\cline{3-8}
\multicolumn{2}{@{\vrule width1pt \ }c|}{ }&\multicolumn{2}{c|}{\Data(2)}&\multicolumn{2}{c|}{ }&\multicolumn{2}{c@{\ \vrule width1pt}}{\multirow{2}{*}{\footnotesize \Data(10)}}\\ 
\cline{1-6}
\multicolumn{2}{@{\vrule width1pt \ }c|}{\multirow{5}{*}{\parbox{4cm}{\small \Data(11)\par \Data(12)から\par \Data(13)\par \Data(14)まで}} }&\multicolumn{2}{c|}{出発地 \Data(15)}&{\tiny 他の交通機関}&泊数&\multicolumn{2}{c@{\ \vrule width1pt}}{ }\\
\cline{3-8}
\multicolumn{2}{@{\vrule width1pt \ }c|}{ }&\multicolumn{2}{l|}{1.\Data(16)}& & &\multirow{2}{*}{\small 本人精算印} &\\
	\cline{3-6}
	\multicolumn{2}{@{\vrule width1pt \ }c|}{ }&\multicolumn{2}{l|}{2.\Data(17)}& & & &\\
	\cline{3-8}
	\multicolumn{2}{@{\vrule width1pt \ }c|}{ }&\multicolumn{2}{l|}{3.\Data(18)}& & &\multicolumn{2}{c|}{精算による過不足}\\
	\cline{3-8}
	\multicolumn{2}{@{\vrule width1pt \ }c|}{ }&\multicolumn{2}{l|}{4.\Data(19)}& & &あり&なし\\
	\hline
	\multicolumn{2}{@{\vrule width1pt \ }c|}{\multirow{2}{*}{精算払・概算払}}&\multicolumn{2}{l|}{5.\Data(20)}& & &\multicolumn{2}{c|}{精算確認年月日}\\
	\cline{3-8}
	\multicolumn{2}{@{\vrule width1pt \ }c|}{ }&\multicolumn{2}{c|}{帰着地 \Data(21)}& & &\multicolumn{2}{c|}{令和 年 月 日}\\
\noalign{\hrule height 1pt}%太い線を引く方法
\end{tabularx}

%旅行期間おわり}}}
\vskip 0.2cm
%備考{{{
\begin{tabularx}{\textwidth}{@{\vrule width1pt \ }>{\centering}m{2.2cm}|X@{\ \vrule width1pt}}
\noalign{\hrule height 1pt}%太い線を引く方法
\multirow{2}{*}{備考}&用務先: \Data(22)\hfill 用務地滞在予定時間: \Data(23) - \Data(24)\hfill  \\
		     &\\
		  &\hfill  \Data(25) \\
\noalign{\hrule height 1pt}%太い線を引く方法
\end{tabularx}
%備考おわり}}}
%\end{table}
\vskip 0.4cm
\dotfill
\vskip 0.4cm
\Large \centering 復  命  書
\vskip 0.2cm
\normalsize
%はんこ{{{
	\begin{tabularx}{\textwidth}{|>{\centering\arraybackslash}X|>{\centering\arraybackslash}X|>{\centering\arraybackslash}X|>{\centering\arraybackslash}X|>{\centering\arraybackslash}X|>{\centering\arraybackslash}m{5cm}|}%6マス
\hline
校   長&教   頭&事務長& &出勤簿&事務処理欄\\
\hline
&&&&&\\
&&&&&\\
&&&&&\\
\hline
\end{tabularx}
%はんこおわり}}}
\vskip 0.2cm
\raggedright
 命により出張しました概要を下記の通り復命します。
\vskip 0.2cm
\hfill 職 教諭   氏名 \Data(2) \ctext{印} 
\vskip 0.2cm
\centering
%報告{{{
\begin{tabularx}{\textwidth}{|@{\vrule width1pt \ }>{\centering}m{2cm}|X@{\ \vrule width1pt}|}
\noalign{\hrule height 1pt}%太い線を引く方法
\multirow{2}{*}{旅行期間}&令和6年  月  日(  )  時  分から\\
&令和6年  月  日(  )  時  分まで\\
\hline
\multirow{3}{*}{報告事項}&\\
	&\\
	&\\
\noalign{\hrule height 1pt}%太い線を引く方法
\end{tabularx}
%報告おわり}}}

\raggedright
\vskip 0.2cm

 ※ 復命書は復校後3日以内に提出してください。
\end{table}

\end{document}
